\documentclass[a4paper]{report}
\usepackage[utf8]{inputenc}
\usepackage[portuguese]{babel}
\usepackage{hyperref}
\usepackage{a4wide}
\hypersetup{pdftitle={PCP - OpenMP},
pdfauthor={João Teixeira, José Ferreira},
colorlinks=true,
urlcolor=blue,
linkcolor=black}
\usepackage{subcaption}
\usepackage{listings}
\usepackage{booktabs}
\usepackage{multirow}
\usepackage{appendix}
\usepackage{tikz}
\usepackage{authblk}
\usepackage{bashful}
\usepackage{verbatim}
\usepackage{amssymb}
\usepackage{multirow}
\usepackage{mwe}
\usepackage[parfill]{parskip}
\usetikzlibrary{positioning,automata,decorations.markings}
\AfterEndEnvironment{figure}{\noindent\ignorespaces}
\AfterEndEnvironment{table}{\noindent\ignorespaces}

\begin{document}

\title{Paradigmas de Computação Paralela\\Bucket Sort com OpenMP}
\author{João Teixeira (A85504) \and José Filipe Ferreira (A83683)}
\date{\today}

\begin{center}
    \begin{minipage}{0.75\linewidth}
        \centering
        \includegraphics[width=0.4\textwidth]{images/eng.jpeg}\par\vspace{1cm}
        \vspace{1.5cm}
        \href{https://www.uminho.pt/PT}
        {\color{black}{\scshape\LARGE Universidade do Minho}} \par
        \vspace{1cm}
        \href{https://www.di.uminho.pt/}
        {\color{black}{\scshape\Large Departamento de Informática}} \par
        \vspace{1.5cm}
        \maketitle
    \end{minipage}
\end{center}

\tableofcontents

\pagebreak

\chapter{Introdução}
O algoritmo escolhido para o projeto da unidade curricular de computação
paralela e distribuída foi o \textit{Bucket Sort}.

Começamos por desenvolver uma versão sequencial do projeto e procedemos ao
\textit{benchmarking} do programa resultante. Em seguida convertemos a
implementação sequencial numa versão com utilização de memoria partilhada
fazendo uso de \textit{OpenMP}.

Ao longo deste relatório iremos descrever a metodologia utilizada e os
resultados de \textit{benchmarking} obtidos ao longo deste projeto.

\chapter{Sequencial}
O \textit{Bucket Sort} consiste em definir um conjunto de N "baldes"
inicialmente vazios. Em seguida os elementos do vetor a ser ordenado são
distribuídos pelos baldes. O critério escolhido para esta distribuição foi
calcular o máximo e o mínimo do vetor a ser ordenado e dividir os intervalos de
valores de cada balde em intervalos do mesmo tamanho. Em seguida o conteúdo de
cada balde é ordenado recorrendo ao \textit{quicksort} presente na
\textit{standard library} de C. Finalmente todos os elementos são copiados um a
um para o vetor original.

Para testar se a primeira implementação sequencial produzia de facto vetores
ordenados criamos um \textit{scrip} que permite gerar N testes aleatórios e
comparar o resultado do nosso programa com o resultado de ordenar os valores
com o comando \textit{sort} de \textit{bash}.

TODO: INSERIR BENCHAMRKS

\chapter{OpenMP}
Fazendo uso da implementação sequencial descrita no capitulo anterior, começamos
a conversão para uma versão com um modelo de memoria partilhada fazendo uso de
\textit{OpenMP}.

Tendo em conta a forma como a escrita em memória era efetuada, e como estamos
perante um paradigma de memória partilhada, a versão inicial do algoritmo foi
rescrita para eliminar a zona crítica na escrita para os \textit{buckets}.
Embora a versão sequencial tenha ficado x\% mais lenta, o algoritmo tornou se
muito mais escalável, visto o elevado custo da gestão de zonas críticas

TODO: INSERIR BENCHAMRKS

\end{document}
